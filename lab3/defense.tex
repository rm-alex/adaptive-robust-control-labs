\documentclass[a4paper,14pt]{extarticle}

\usepackage[T2A]{fontenc}
\usepackage[utf8]{inputenc}
\usepackage[english, russian]{babel}

\usepackage[left=30mm, right=10mm, top=20mm, bottom=20mm]{geometry}

\usepackage{tempora}
\usepackage{setspace}
\onehalfspacing

\usepackage{titlesec}
\titleformat{\section}[block]{\bfseries\centering\MakeUppercase}{\thesection.}{1em}{}
\titleformat{\subsection}[block]{\bfseries}{\thesubsection.}{1em}{}
\titleformat{\subsubsection}[block]{\bfseries}{\thesubsubsection.}{1em}{}

\renewcommand{\contentsname}{\hfill \textbf{СОДЕРЖАНИЕ} \hfill\null}

\usepackage{indentfirst}
\setlength{\parindent}{1.25cm}

\usepackage{amsmath, amsfonts, amssymb}
\usepackage{graphicx}
\usepackage{caption}
\usepackage{subcaption}
\usepackage{float}
\usepackage{tikz}
\usetikzlibrary{patterns}
\usepackage{cmap}
\usepackage{hyperref}
\usepackage{xcolor}
\usepackage{listings}

\definecolor{LightGray}{gray}{0.7}

\lstdefinestyle{code}{
    language=Python, % change if needed
    basicstyle=\small\ttfamily,
    numbers=left,
    numberstyle=\small\color{LightGray},
    stepnumber=1,
    numbersep=5pt,
    backgroundcolor=\color{white},
    showspaces=false,
    showstringspaces=false,
    showtabs=false,
    tabsize=4,
    captionpos=b,
    breaklines=true,
    breakatwhitespace=false,
    frame=single,
    rulecolor=\color{LightGray},
    linewidth=\linewidth,
    keywordstyle=\color{blue}\bfseries,
    commentstyle=\color{green!40!black},
    stringstyle=\color{violet},
    escapeinside={\%*}{*)},
    xleftmargin=10pt,
    xrightmargin=10pt,
    framexleftmargin=0pt,
    framexrightmargin=0pt
}
\lstset{style=code}

\hypersetup{
    colorlinks=true,
    linkcolor=blue,
    filecolor=magenta,
    urlcolor=cyan,
    pdftitle={def},
    pdfauthor={Rumyantsev Alexey},
    pdfsubject={control},
    pdfkeywords={LaTeX, PDF},
    pdfpagemode=FullScreen,
}

\graphicspath{{src/images/}}

\begin{document}
\section{Задание 1}
\subsection{Условие}
Решить задачу адаптивного слежения для объекта вида
(вывести аналитически выражения закона управления и алгоритма адаптации):
$$
\dot{x}=\theta_1\cos{x}-\theta_2\cos{\left( x^4 \right)}+\theta_3\sin{\left(2x+9\right)}+\theta_2x^3+\cos{\left( x^6 \right)}+3u,
$$
где $\theta_i$ -- неизвестный параметр. Цель управления заключается
в компенсации неопределенности $\theta_i$ и обеспечении следующего
целевого равенства:
$$
\lim\limits_{t\to\infty}\left( x_m(t)-x(t) \right)=\lim\limits_{t\to\infty}\varepsilon(t)=0,
$$
где $\varepsilon=x_m-x$ -- ошибка управления,
$x_m$ -- эталонный сигнал, являющийся выходом
динамической модели (эталонной модели):
$$
\dot{x}_m=-\lambda x_m+\lambda g,
$$
где $g$ -- сигнал задания, $\lambda>0$ -- параметр,
задающий время переходного процесса.


\subsection{Решение}
Запишем уравнение объекта в регрессионной форме:
$$
\dot{x}=\begin{bmatrix}
    \cos{x}\\ x^3-\cos{\left( x^4 \right)}\\ \sin{\left( 2x+9 \right)}
\end{bmatrix}^T\begin{bmatrix}
    \theta_1\\ \theta_2\\ \theta_3
\end{bmatrix}+\cos{\left( x^6 \right)}+3u=f(x)^T\theta+d(x)+3u,
$$
$$
f(x)=\begin{bmatrix}
    f_1(x)\\f_2(x)\\f_3(x)
\end{bmatrix},\ \theta=\begin{bmatrix}
    \theta_1\\ \theta_2\\ \theta_3
\end{bmatrix},\ d(x)=\cos{\left( x^6 \right)}
$$


Ошибка слежения:
$$
\varepsilon(t)=x_m(t)-x(t)\Rightarrow\dot{\varepsilon}=\dot{x}_m-\dot{x}
$$


Подставим $\dot{x}_m$ и $\dot{x}$:
\begin{align*}
    \dot{\varepsilon}=&-\lambda x_m+\lambda g -\left( \theta_1\cos{x}-\theta_2\cos{\left( x^4 \right)}+\right.\\
    &\left.+\theta_3\sin{\left(2x+9\right)}+\theta_2x^3+\cos{\left( x^6 \right)}+3u \right)=\\
    =&-\lambda x_m+\lambda g-\theta_1\cos{x}-\theta_2\left( x^3-\cos{\left( x^4 \right)} \right)+\\
    &-\theta_3\sin{\left(2x+9\right)}-\cos{\left( x^6 \right)}-3u=\\
    =&-\lambda x_m+\lambda g-f(x)^T\theta-d(x)-3u
\end{align*}


Подставим $x_m=x+\varepsilon$ в $\dot{\varepsilon}$:
$$
\dot{\varepsilon}=-\lambda \left( x+\varepsilon \right)+\lambda g-f(x)^T\theta-d(x)-3u,
$$
$$
\dot{\varepsilon}=-\lambda\varepsilon-\left( \lambda x-\lambda g+f(x)^T\theta+d(x)+3u \right)
$$


Желаемая динамика ошибки $\varepsilon\to0$:
$$
\dot{\varepsilon}=-k\varepsilon,\ k>0
$$


Зададим такой закон управления, чтобы часть
выражения $\dot{\varepsilon}$ в скобках обнулилась:
$$
3u=-\lambda x+\lambda g-f(x)^T\theta-d(x)\Rightarrow u=\frac{1}{3}\left( -\lambda x+\lambda g-f(x)^T\theta-d(x) \right)
$$


Так как $\theta$ неизвестна, зададим:
$$
\hat{\theta}=\begin{bmatrix}
    \hat{\theta}_1\\ \hat{\theta}_2\\ \hat{\theta}_3
\end{bmatrix}
$$


Тогда, адаптивный закон управления:
$$
u=\frac{1}{3}\left( -\lambda x+\lambda g-f(x)^T\hat{\theta}-d(x) \right)
$$


Определим вектор ошибки параметров:
$$
\tilde{\theta}=\theta-\hat{\theta}
$$


Подставим $u$ в $\dot{x}$:
$$
\dot{x}=f(x)^T\theta+d(x)+\left( -\lambda x+\lambda g-f(x)^T\hat{\theta}-d(x) \right)=-\lambda x+\lambda g+f(x)^T\tilde{\theta}
$$


Тогда:
$$
\dot{\varepsilon}=\dot{x}_m-\dot{x}=-\lambda x_m+\lambda g-\left( -\lambda x+\lambda g+f(x)^T\tilde{\theta} \right)=-\lambda \varepsilon -f(x)^T\tilde{\theta}
$$


Квадратичная функция Ляпунова:
$$
V=0.5\varepsilon^2+0.5\tilde{\theta}^TP^{-1}\tilde{\theta},
$$
где $P=P^T\succ0$ -- матрица скорости адаптации:
$$
P=\begin{bmatrix}
    p_1&0&0\\ 0&p_2&0\\0&0&p_3
\end{bmatrix},\ p_i>0
$$


Производная функции Ляпунова:
$$
\dot{V}=\varepsilon\dot{\varepsilon}+\tilde{\theta}^TP^{-1}\dot{\tilde{\theta}}
$$


Так как $\theta=const.$:
$$
\dot{\tilde{\theta}}=0-\dot{\hat{\theta}}=-\dot{\hat{\theta}}
$$


Подставим $\dot{\varepsilon}$ и $\dot{\tilde{\theta}}$ в $\dot{V}$:
$$
\dot{V}=\varepsilon\left( -\lambda\varepsilon-f(x)^T\tilde{\theta} \right)-\tilde{\theta}^TP^{-1}\dot{\hat{\theta}}=-\lambda\varepsilon^2-\varepsilon f(x)^T\tilde{\theta}-\tilde{\theta}^TP^{-1}\dot{\hat{\theta}}
$$


Так как $\varepsilon f(x)^T\tilde{\theta}=\tilde{\theta}^T\left( \varepsilon f(x)\right)$:
$$
\dot{V}=-\lambda\varepsilon^2-\tilde{\theta}^T\left( \varepsilon f(x)+P^{-1}\dot{\hat{\theta}} \right)
$$


Для $\dot{V}\leq-\lambda\varepsilon^2\leq0$ достаточно:
$$
\varepsilon f(x)+P^{-1}\dot{\hat{\theta}}=0\Rightarrow \dot{\hat{\theta}}=-P\varepsilon f(x)
$$


Тогда, алгоритм адаптации:
$$
\begin{cases}
    \dot{\hat{\theta}}_1=-p_1\varepsilon\cos{x},\\
    \dot{\hat{\theta}}_2=-p_2\varepsilon\left( x^3-\cos{\left( x^4 \right)} \right),\\
    \dot{\hat{\theta}}_3=-p_3\varepsilon\sin{\left( 2x+9 \right)}
\end{cases}
$$


Закон управления:
$$
u=\frac{1}{3}\left( -\lambda x+\lambda g -\hat{\theta}_1\cos{x}-\hat{\theta}_2\left( x^3-\cos{\left( x^4 \right)} \right)-\hat{\theta}_3\sin{\left( 2x+9 \right)}-\cos{\left( x^6 \right)} \right)
$$


\section{Задание 2}
\subsection{Условие}
Решить задачу адаптивного управления
(вывести аналитически выражения закона управления и алгоритма адаптации)
для объекта:
$$
\begin{cases}
    \dot{x}_1=&x_2,\\
    \dot{x}_2=&\theta_1\cos{\left( x_1^2-x_2^5 \right)}+\theta_2\sin{\left( 6x_1^2-7x_2 \right)}+\\
    &+\theta_3\sin{\left( x_2^2 \right)}-\theta_2x_2^5+\cos{x_3}+6u,
\end{cases}
$$
где $\theta_i$ -- неизвестные коэффициенты. Цель управления задается
равенством:
$$
\lim\limits_{t\to\infty}||e(t)||=0,
$$
где $e=x_m-x$ -- вектор ошибки управления,
$x_m\in\mathbb{R}^n$ -- вектор, генерируемый эталонной моделью:
$$
\dot{x}_m=A_mx_m+b_mg
$$
с задающим воздействием $g(t)$.


\subsection{Выполнение}
Обозначим:
$$
\theta=\begin{bmatrix}
    \theta_1\\ \theta_2\\ \theta_3
\end{bmatrix},\ f(x)=\begin{bmatrix}
    \cos{\left( x_1^2-x_2^5 \right)}\\
    \sin{\left( 6x_1-7x_2 \right)}-x_2^5\\
    \sin{\left( x_2^2 \right)}
\end{bmatrix},\ d(x)=\cos{x_3}
$$


Система:
$$
\begin{cases}
    \dot{x}_1=x_2,\\
    \dot{x}_2=\theta^Tf(x)+d(x)+6u
\end{cases}
$$


Рассмотрим ошибку:
$$
e=x_1-x=\begin{bmatrix}
    e_1\\e_2
\end{bmatrix}\Rightarrow \dot{e}=\dot{x}_m-\dot{x},
$$
$$\dot{e}=A_mx_m+b_mg-\left(\begin{bmatrix}
    x_2\\ \theta^Tf(x)
\end{bmatrix}+\begin{bmatrix}0\\1\end{bmatrix}d(x)+\begin{bmatrix}
    0\\6
\end{bmatrix}u\right)
$$


Обозначим:
$$
\begin{bmatrix}
    0\\1
\end{bmatrix}d(x)=B_dd(x),\ \begin{bmatrix}
    0\\6
\end{bmatrix}u=Bu
$$


Подставим $x_m=e+x$:
$$
\dot{e}=A_m\left( e+x \right)+b_mg-\left( \begin{bmatrix}
    x_2\\ \theta^Tf(x)
\end{bmatrix}+B_dd(x)+Bu \right),
$$
$$
\dot{e}=A_me+b_mg+A_mx-\begin{bmatrix}
    x_2\\ \theta^Tf(x)
\end{bmatrix}-B_dd(x)-Bu
$$


Цель:
$$
\dot{e}=A_me
$$


Тогда:
$$
Bu=b_mg+A_mx-\begin{bmatrix}
    x_2\\ \theta^Tf(x)
\end{bmatrix}-B_dd(x)
$$


Введем матрицы:
$$
A_m=\begin{bmatrix}
    0&1\\ a_{m_{21}}&a_{m_{22}}
\end{bmatrix},\ b_m=\begin{bmatrix}
    0\\ b_{m_0}
\end{bmatrix}
$$


Итого:
$$
6u=a_{m_{21}}x_{1}+a_{m_{22}}x_{2}+b_{m_0}g-\theta^Tf(x)-d(x),
$$
$$
u=\frac{1}{6}\left( a_{m_{21}}x_{1}+a_{m_{22}}x_{2}+b_{m_0}g-\theta^Tf(x)-d(x) \right)
$$


Так как $\theta$ неизвестна:
$$
\hat{\theta}=\begin{bmatrix}
    \hat{\theta}_1\\ \hat{\theta}_2\\ \hat{\theta}_3
\end{bmatrix},\ \tilde{\theta}=\theta-\hat{\theta}
$$


Управление:
$$
u=\frac{1}{6}\left( a_{m_{21}}x_{1}+a_{m_{22}}x_{2}+b_{m_0}g-\hat{\theta}^Tf(x)-d(x) \right)
$$


Подставим в $\dot{e}$:
$$
\begin{cases}
    \dot{e}_1=e_2,\\
    \dot{e}_2=a_{m_{21}}e_{1}+a_{m_{22}}e_{2}-\theta^Tf(x)+\hat{\theta}^Tf(x)
\end{cases}
$$


Так как $-\left( \theta^T-\hat{\theta}^T \right)f(x)=-\tilde{\theta}^Tf(x)$:
$$
\dot{e}=A_me-B\tilde{\theta}^Tf(x)
$$


Функция Ляпунова:
$$
V(e,\tilde{\theta})=0.5e^TP_1e+0.5\tilde{\theta}^TP_2^{-1}\tilde{\theta},
$$
где $P_1=P_1^T\succ0$ -- решение уравнения Ляпунова:
$$
A_m^TP_1+P_1A_m=-Q,\ Q=Q^T\succ0,
$$
$P_2$ -- диагональная матрица скоростей адаптации:
$$
P_2=\begin{bmatrix}
    p_1&0&0\\0&p_2&0\\0&0&p_3
\end{bmatrix},\ p_i>0
$$


Ее производная:
$$
\dot{V}=e^TP_1\dot{e}+\tilde{\theta}^TP_2^{-1}\dot{\tilde{\theta}}
$$


Подставим $\dot{e}$:
$$
\dot{V}=e^TP_1\left( A_me-B\tilde{\theta}^Tf(x) \right)+\tilde{\theta}^TP_2^{-1}\dot{\tilde{\theta}}=
e^TP_1A_me-e^TP_1B\tilde{\theta}^Tf(x)+\tilde{\theta}^TP_2^{-1}\dot{\tilde{\theta}}
$$


Рассмотрим первый член:
$$
e^TP_1A_me=0.5e^T\left( P_1A_m+A_m^TP_1 \right)e=-0.5e^TQe
$$


Рассмотрим второй член:
$$
-e^TP_1B\tilde{\theta}^Tf(x)=-\tilde{\theta}^T\left( e^TP_1B \right)f(x)
$$


Тогда:
$$
\dot{V}=-0.5e^TQe-\tilde{\theta}^T\left( e^TP_1B \right)f(x)+\tilde{\theta}^TP_2^{-1}\dot{\tilde{\theta}}
$$


$\dot{\tilde{\theta}}=-\dot{\hat{\theta}}$, т.к. $\theta=const.$:
$$
\dot{V}=-0.5e^TQe-\tilde{\theta}^T\left( e^TP_1B \right)f(x)-\tilde{\theta}^TP_2^{-1}\dot{\hat{\theta}},
$$
$$
\dot{V}=-0.5e^TQe-\tilde{\theta}^T\left(  \left( e^TP_1B \right)f(x) +P_2^{-1}\dot{\hat{\theta}}\right)
$$


Для $\dot{V}=-0.5e^TQe\leq0$ достаточно:
$$
\left( e^TP_1B \right)f(x) +P_2^{-1}\dot{\hat{\theta}}=0,
$$
$$
\dot{\hat{\theta}}=-P_2\left( e^TP_1B \right)f(x)
$$


То есть:
$$
\begin{cases}
    \dot{\hat{\theta}}_1=-p_1\left( e^TP_1B \right)\cos{\left( x_1^2-x_2^5 \right)},\\
    \dot{\hat{\theta}}_2=-p_2\left( e^TP_1B \right)\left( \sin{\left( 6x_1^2-7x_2 \right)}-x_2^5 \right),\\
    \dot{\hat{\theta}}_3=-p_3\left( e^TP_1B \right)\sin{\left( x_2^2 \right)}
\end{cases}
$$


Закон управления:
\begin{align*}
    u=\frac{1}{6}&\left( a_{m_{21}}x_{1}+a_{m_{22}}x_{2}+b_{m_0}g-\theta_1\cos{\left( x_1^2-x_2^5 \right)}+\right.\\
    &\left.-\theta_2\left(\sin{\left( 6x_1^2-7x_2 \right)}-x_2^5\right)-\theta_3\sin{\left( x_2^2 \right)}-\cos{x_3} \right)
\end{align*}
\end{document}