\documentclass[a4paper,14pt]{extarticle}

\usepackage[T2A]{fontenc}
\usepackage[utf8]{inputenc}
\usepackage[english, russian]{babel}

\usepackage[left=30mm, right=10mm, top=20mm, bottom=20mm]{geometry}

\usepackage{tempora}
\usepackage{setspace}
\onehalfspacing

\usepackage{titlesec}
\titleformat{\section}[block]{\bfseries\centering\MakeUppercase}{\thesection.}{1em}{}
\titleformat{\subsection}[block]{\bfseries}{\thesubsection.}{1em}{}
\titleformat{\subsubsection}[block]{\bfseries}{\thesubsubsection.}{1em}{}

\renewcommand{\contentsname}{\hfill \textbf{СОДЕРЖАНИЕ} \hfill\null}

\usepackage{indentfirst}
\setlength{\parindent}{1.25cm}

\usepackage{amsmath, amsfonts, amssymb}
\usepackage{graphicx}
\usepackage{caption}
\usepackage{subcaption}
\usepackage{float}
\usepackage{tikz}
\usetikzlibrary{patterns}
\usepackage{cmap}
\usepackage{hyperref}
\usepackage{xcolor}
\usepackage{listings}

\definecolor{LightGray}{gray}{0.7}

\lstdefinestyle{code}{
    language=matlab, % change if needed
    basicstyle=\small\ttfamily,
    numbers=left,
    numberstyle=\small\color{LightGray},
    stepnumber=1,
    numbersep=5pt,
    backgroundcolor=\color{white},
    showspaces=false,
    showstringspaces=false,
    showtabs=false,
    tabsize=4,
    captionpos=b,
    breaklines=true,
    breakatwhitespace=false,
    frame=single,
    rulecolor=\color{LightGray},
    linewidth=\linewidth,
    keywordstyle=\color{blue}\bfseries,
    commentstyle=\color{green!40!black},
    stringstyle=\color{violet},
    escapeinside={\%*}{*)},
    xleftmargin=10pt,
    xrightmargin=10pt,
    framexleftmargin=0pt,
    framexrightmargin=0pt
}
\lstset{style=code}

\hypersetup{
    colorlinks=true,
    linkcolor=blue,
    filecolor=magenta,
    urlcolor=cyan,
    pdftitle={lab arc},
    pdfauthor={Rumyantsev Alexey},
    pdfsubject={control},
    pdfkeywords={LaTeX, PDF},
    pdfpagemode=FullScreen,
}

\graphicspath{{src/images/}}

\begin{document}

\begin{titlepage}
    \begin{center}
        МИНИСТЕРСТВО НАУКИ И ВЫСШЕГО ОБРАЗОВАНИЯ РОССИЙСКОЙ ФЕДЕРАЦИИ\\
        \vspace*{2.5mm}
        Федеральное государственное автономное образовательное учреждение высшего образования
        «НАЦИОНАЛЬНЫЙ ИССЛЕДОВАТЕЛЬСКИЙ УНИВЕРСИТЕТ ИТМО»\\
        \vspace*{2.5mm}
        ФАКУЛЬТЕТ СИСТЕМ УПРАВЛЕНИЯ И РОБОТОТЕХНИКИ
        \vfill

        {\large\bfseries ОТЧЕТ ПО ЛАБОРАТОРНОЙ РАБОТЕ №6}\\
        {\large по дисциплине}\\
        {\large«АДАПТИВНОЕ И РОБАСТНОЕ УПРАВЛЕНИЕ»}\\
        {\large на тему}\\
        {\large «СИНТЕЗ АДАПТИВНОГО НАБЛЮДАТЕЛЯ СОСТОЯНИЯ ЛИНЕЙНОГО ОБЪЕКТА»}\\
        Вариант 21
        \vfill

        \begin{flushright}
            Выполнили: студенты\\
            Дьячихин Д. Н., R3480\\
            Румянцев А. А., R3441\medskip\\

            Проверил: преподаватель\\
            Парамонов А. В.
        \end{flushright}
        \vfill

        Санкт-Петербург\\
        2025
    \end{center}
\end{titlepage}

\setcounter{page}{2}
\tableofcontents
\newpage

\section{Цель работы}
Освоение процедуры синтеза адаптивного наблюдателя линейного объекта.


\section{Постановка задачи}
Дан асимптотически устойчивый объект управления:
\begin{align}
    \begin{cases}
        \dot{x}=Ax+Bu,\ x(0),\\
        y=Cx,
    \end{cases}\label{syseq:1}
\end{align}
где $x$ -- недоступный прямому измерению вектор состояния,
$u,y$ -- входной и выходной сигналы объекта, доступные прямым измерениям,
$$
A=\begin{bmatrix}
    -a_{n-1} &1 &\hdots &0\\
    -a_{n-2} &0 &\hdots &0\\
    a_{n-1} &\vdots &\ddots &1\\
     -a_{0} &0 &\hdots &0
\end{bmatrix},\ b=\begin{bmatrix}
    0\\ \vdots\\0\\b_m\\\vdots\\b_0
\end{bmatrix},\ C=\begin{bmatrix}
    1&0&\hdots&0
\end{bmatrix},
$$
$a_i,i=\overline{0,n-1},b_j,j=\overline{0,m}$ --
неизвестные коэффициенты модели.


Рассматриваемая задача состоит в построении оценки вектора
состояния $\hat{x}$ такой, что:
\begin{align}
    \lim\limits_{t\to\infty}||x(t)-\hat{x}(t)||=0\label{eq:aim}
\end{align}


Синтезируемый адаптивный наблюдатель должен одновременно
оценивать неизвестные параметры объекта управления $\theta$
и генерировать оценку вектора состояния $\hat{x}$.


Отметим, что в задаче класс объектов (\ref{syseq:1}) ограничен следующим
допущением (условием согласования): для некоторого $n$-мерного
вектора $\bar{\theta}$ матрицы $A,C,A_0$
связаны следующим соотношением:
$$
A_0=A-\bar{\theta}C
$$
Можно показать, что для рассматриваемого класса объектов:
$$
\bar{\theta}=\begin{bmatrix}
    k_{n-1}-a_{n-1}\\
    k_{n-2}-a_{n-2}\\
    \vdots\\
    k_0-a_0
\end{bmatrix}
$$


\section{Теоретическая часть}
Для решения задачи используется
параметризованное представление выходной переменной:
\begin{align}
    y=\theta^T\omega\label{eq:y},
\end{align}
где $\omega$ -- вектор регрессоров, состоящий из двух фильтрованных через $A_0$ компонент выхода $y$ и двух фильтрованных компонент входа $u$, и вектора состояния:
\begin{align}
    x=\sum\limits_{i=0}^{n-1}\theta_{i+1}\left( Is-A_0 \right)^{-1}
    e_{n-i}[y]+\sum\limits_{j=0}^{m}\theta_{j+1+n}\left( Is-A_0 \right)^{-1}
    e_{n-j}[u]\label{eq:x}
\end{align}
Заменим в (\ref{eq:y}) параметры $\theta$
на оценки $\hat{\theta}$ и сформируем
настраиваемую модель объекта:
\begin{align}
    \hat{y}=\hat{\theta}^T\omega,\label{eq:hat_y}
\end{align}
где $\hat{y}$ -- оценка переменной $y$.
Введем в рассмотрение ошибку
идентификации:
$$
\varepsilon=y-\hat{y}
$$
Учитывая (\ref{eq:y}), (\ref{eq:hat_y}), получаем:
\begin{align}
    \varepsilon=\tilde{\theta}^T\omega,\label{eq:vare}
\end{align}
где $\tilde{\theta}=\theta-\hat{\theta}$ --
вектор параметрических ошибок. Последнее выражение
представляет собой стандартную статическую модель ошибок,
на основе которой при помощи функции Ляпунова $V=\tilde{\theta}^T\tilde{\theta}/2\gamma$
и анализа ее производной строится алгоритм адаптации:
\begin{align}
    \dot{\hat{\theta}}=\gamma\omega\varepsilon,\label{eq:dht}
\end{align}
где $\gamma>0$ -- коэффициент адаптации.


Действительно, расчет производной $\dot{V}$ дает:
$$
\dot{V}=\frac{1}{\gamma}\tilde{\theta}^T\dot{\tilde{\theta}}=-\frac{1}{\gamma}\tilde{\theta}^T\dot{\hat{\theta}}
$$
При выборе структуры алгоритма адаптации (\ref{eq:dht}) имеем:
$$
\dot{V}=-\frac{1}{\gamma}\tilde{\theta}^T\gamma\omega\varepsilon=-\varepsilon^2<0
$$
Из последнего неравенства при условии ограниченности функции $\omega$ и ее
первой производной $\dot{\omega}$ (условие накладывается на входной сигнал $u$)
следуют свойства системы, состоящей (\ref{eq:vare}), (\ref{eq:dht}):
\begin{itemize}
    \item все сигналы в системе ограничены;
    \item ошибка $\varepsilon$ стремится к нулю асимптотически;
    \item параметрические ошибки $\tilde{\theta}$
    стремятся к нулю экспоненциально,
    если вектор $\omega$ удовлетворяет условию неисчезающего возбуждения:
    \begin{align}
        \int\limits_{t}^{t+T}\omega(\tau)\omega^T\,d\tau>\alpha I,\ \alpha>0,T>0\label{eq:unv}
    \end{align}
    Условие (\ref{eq:unv}) в конечном итоге зависит от
    частотной насыщенности сигнала $u$, который должен содержать
    <<достаточное>> количество гармоник;
    \item если ошибки $\tilde{\theta}$ стремятся к нулю,
    то оценка вектора состояния $\hat{x}$ также стремится к $x$.
\end{itemize}


После замены в (\ref{eq:x}) параметров $\theta$ на оценки $\hat{\theta}$ получаем оценку
вектора состояния:
\begin{align}
    \hat{x}=\sum\limits_{i=0}^{n-1}\hat{\theta}_{i+1}\left( Is-A_0 \right)^{-1}
    e_{n-i}[y]+\sum\limits_{j=0}^{m}\hat{\theta}_{j+1+n}\left( Is-A_0 \right)^{-1}
    e_{n-j}[u]\label{eq:hat_x}
\end{align}


Таким образом, адаптивный наблюдатель, обеспечивающий
выполнение условия (\ref{eq:aim}) (при выполнении условия неисчезающего
возбуждения (\ref{eq:unv})), состоит из настраиваемой модели (\ref{eq:hat_y}), алгоритма
адаптации (\ref{eq:dht}) и алгоритма оценивания вектора состояния (\ref{eq:hat_x}).


\section{Экспериментальная часть}
\subsection{Определение параметров системы}
Согласно варианту 21, в данной работе система (\ref{syseq:1})
имеет вид:
$$
A=\begin{bmatrix}
    -a_1&1\\-a_0&0
\end{bmatrix}=\begin{bmatrix}
    -1&1\\-5&0
\end{bmatrix},\ b=\begin{bmatrix}
    b_1\\b_0
\end{bmatrix}=\begin{bmatrix}
    3\\9
\end{bmatrix},\ C=\begin{bmatrix}
    1&0
\end{bmatrix},
$$


Гурвицева матрица с коэффициентами фильтра:
$$
A_0=\begin{bmatrix}
    -k_1&1\\-k_0&0
\end{bmatrix}=\begin{bmatrix}
    -8&1\\-16&0
\end{bmatrix}
$$


Вектор параметров:
$$
\theta=\begin{bmatrix}
    k_0-a_0\\k_1-a_1\\b_0\\b_1
\end{bmatrix}=\begin{bmatrix}
    11\\7\\9\\3
\end{bmatrix}
$$


\subsection{Моделирование адаптивного наблюдателя при синусоидальном входном сигнале}
Промоделируем адаптивный наблюдатель вектора состояния объекта (\ref{eq:hat_y}),
(\ref{eq:dht}), (\ref{eq:hat_x}) при входном сигнале:
$$
u(t)=10\sin{t}:=u_1
$$
и коэффициенте адаптации $\gamma=1$.


Схема моделирования:
\begin{figure}[H]
    \centering
    \includegraphics[scale=0.43]{scheme.png}
    \caption{Схема моделирования}
    \label{fig:sch}
\end{figure}


Код блока матлаб-функции сигнала $u_1$:
\begin{lstlisting}[label=c1, caption={Реализация матлаб-функции $u_1$}]
function u = fcn(t)
u = 10*sin(t);
\end{lstlisting}


Код блока матлаб-функции, расcчитывающий $\hat{y},\varepsilon,\tilde{\theta}$ и $\dot{\hat{\theta}}$:
\begin{lstlisting}[label=cc1, caption={Реализация матлаб-функции расчета $\hat{y},\varepsilon,\tilde{\theta}$ и $\dot{\hat{\theta}}$}]
function [y_hat, eps, theta_tilde, theta_hat_dot] = fcn(y, omega, theta_hat, theta)
gamma=1;
y_hat = theta_hat'*omega;
eps = y-y_hat;
theta_tilde = theta-theta_hat;
theta_hat_dot = gamma*omega*eps;
\end{lstlisting}


Исходные данные схемы:
\begin{lstlisting}[label=ccc1, caption={Исходные данные схемы симулинк: $a_i,b_i,k_i,\theta,A_0,A,b,C,D$}]
k0=16;
k1=8;
a0=5;
a1=1;
b0=9;
b1=3;
theta=[k0-a0;k1-a1;b0;b1];
A0=[-k1 1; -k0 0];
A=[-a1 1; -a0 0];
b=[b1;b0];
C=[1 0];
D=[0;0];
\end{lstlisting}


Выполним моделирование:
\begin{figure}[H]
    \centering
    \includegraphics[scale=1]{e1.png}
    \caption{График нормы ошибки $||x(t)-\hat{x}(t)||$ при $u_1$}
    \label{fig:e1}
\end{figure}
\begin{figure}[H]
    \centering
    \includegraphics[scale=1]{tt1.png}
    \caption{График параметрической ошибки $\tilde{\theta}$ при $u_1$}
    \label{fig:tt1}
\end{figure}


\subsection{Моделирование адаптивного наблюдателя при входном сигнале, состоящем из суммы гармоник}
Повторим эксперимент при:
$$
u(t)=10\sin{t}+5\cos{2t}+4\cos{4t}+3\cos{8t}:=u_2
$$


Код блока матлаб-функции сигнала $u_2$:
\begin{lstlisting}[label=c2, caption={Реализация матлаб-функции $u_2$}]
function u = fcn(t)
u = 10*sin(t)+5*cos(2*t)+4*cos(4*t)+3*cos(8*t);
\end{lstlisting}


Выполним моделирование:
\begin{figure}[H]
    \centering
    \includegraphics[scale=1]{e2.png}
    \caption{График нормы ошибки $||x(t)-\hat{x}(t)||$ при $u_2$}
    \label{fig:e2}
\end{figure}
\begin{figure}[H]
    \centering
    \includegraphics[scale=1]{tt2.png}
    \caption{График параметрической ошибки $\tilde{\theta}$ при $u_2$}
    \label{fig:tt2}
\end{figure}


\subsection{Выводы}
При входном сигнале в виде одной гармоники $u_1$
оценки состояния и параметров не сходятся с нулевой ошибкой к
истинным значениям. Вместо этого получается устоявшаяся ошибка.
Это можно объяснить тем, что условие неисчезающего возбуждения
для $\omega$ при таком входе не выполняется -- входной сигнал
недостаточно разнообразен.


При входном сигнале из нескольких гармоник $u_2$
условие неисчезающего возбуждения для $\omega$ выполняется, ошибки
оценок стремятся к нулю асимптотически. Это подтверждает
корректность синтеза адаптивного наблюдателя.


\section{Вывод}
В ходе выполнения лабораторной работы
был смоделирован адаптивный наблюдатель
при различных входных сигналах.
Результаты показали, что при невыполнении
условия неисчезающего возбуждения для вектора регрессоров ошибки
оценивания не стремятся асимптотически к нулю,
а сходятся с устоявшейся ошибкой.
\end{document}