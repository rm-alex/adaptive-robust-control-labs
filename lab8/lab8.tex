\documentclass[a4paper,14pt]{extarticle}

\usepackage[T2A]{fontenc}
\usepackage[utf8]{inputenc}
\usepackage[english, russian]{babel}

\usepackage[left=30mm, right=10mm, top=20mm, bottom=20mm]{geometry}

\usepackage{tempora}
\usepackage{setspace}
\onehalfspacing

\usepackage{titlesec}
\titleformat{\section}[block]{\bfseries\centering\MakeUppercase}{\thesection.}{1em}{}
\titleformat{\subsection}[block]{\bfseries}{\thesubsection.}{1em}{}
\titleformat{\subsubsection}[block]{\bfseries}{\thesubsubsection.}{1em}{}

\renewcommand{\contentsname}{\hfill \textbf{СОДЕРЖАНИЕ} \hfill\null}

\usepackage{indentfirst}
\setlength{\parindent}{1.25cm}

\usepackage{amsmath, amsfonts, amssymb}
\usepackage{graphicx}
\usepackage{caption}
\usepackage{subcaption}
\usepackage{float}
\usepackage{tikz}
\usetikzlibrary{patterns}
\usepackage{cmap}
\usepackage{hyperref}
\usepackage{xcolor}
\usepackage{listings}

\definecolor{LightGray}{gray}{0.7}

\lstdefinestyle{code}{
    language=matlab, % change if needed
    basicstyle=\small\ttfamily,
    numbers=left,
    numberstyle=\small\color{LightGray},
    stepnumber=1,
    numbersep=5pt,
    backgroundcolor=\color{white},
    showspaces=false,
    showstringspaces=false,
    showtabs=false,
    tabsize=4,
    captionpos=b,
    breaklines=true,
    breakatwhitespace=false,
    frame=single,
    rulecolor=\color{LightGray},
    linewidth=\linewidth,
    keywordstyle=\color{blue}\bfseries,
    commentstyle=\color{green!40!black},
    stringstyle=\color{violet},
    escapeinside={\%*}{*)},
    xleftmargin=10pt,
    xrightmargin=10pt,
    framexleftmargin=0pt,
    framexrightmargin=0pt
}
\lstset{style=code}

\hypersetup{
    colorlinks=true,
    linkcolor=blue,
    filecolor=magenta,
    urlcolor=cyan,
    pdftitle={lab arc},
    pdfauthor={Rumyantsev Alexey},
    pdfsubject={control},
    pdfkeywords={LaTeX, PDF},
    pdfpagemode=FullScreen,
}

\graphicspath{{src/images/}}

\begin{document}

\begin{titlepage}
    \begin{center}
        МИНИСТЕРСТВО НАУКИ И ВЫСШЕГО ОБРАЗОВАНИЯ РОССИЙСКОЙ ФЕДЕРАЦИИ\\
        \vspace*{2.5mm}
        Федеральное государственное автономное образовательное учреждение высшего образования
        «НАЦИОНАЛЬНЫЙ ИССЛЕДОВАТЕЛЬСКИЙ УНИВЕРСИТЕТ ИТМО»\\
        \vspace*{2.5mm}
        ФАКУЛЬТЕТ СИСТЕМ УПРАВЛЕНИЯ И РОБОТОТЕХНИКИ
        \vfill

        {\large\bfseries ОТЧЕТ ПО ЛАБОРАТОРНОЙ РАБОТЕ №8}\\
        {\large по дисциплине}\\
        {\large«АДАПТИВНОЕ И РОБАСТНОЕ УПРАВЛЕНИЕ»}\\
        {\large на тему}\\
        {\large «АДАПТИВНОЕ УПРАВЛЕНИЕ ЛИНЕЙНЫМ
                ОБЪЕКТОМ ПО ВЫХОДУ НА ОСНОВЕ АЛГОРИТМА АДАПТАЦИИ С
                РАСШИРЕННОЙ ОШИБКОЙ»}\\
        Вариант 21
        \vfill

        \begin{flushright}
            Выполнили: студенты\\
            Дьячихин Д. Н., R3480\\
            Румянцев А. А., R3441\medskip\\

            Проверил: преподаватель\\
            Парамонов А. В.
        \end{flushright}
        \vfill

        Санкт-Петербург\\
        2025
    \end{center}
\end{titlepage}

\setcounter{page}{2}
\tableofcontents
\newpage

\section{Цель работы}
Освоение метода расширенной ошибки в задачах
адаптивного управления по выходу.


\section{Постановка задачи}
Рассмотрим минимально-фазовую
линейную модель объекта,
представленную в форме <<вход-выход>>:
\begin{align}
    y^{(n)}+a_{n-1}y^{(n-1)}+...+a_0y=b_mu^{(m)}+b_{m-1}u^{(m-1)}+...+b_0u,\label{eq:mpomio}
\end{align}
где $a_i,i=\overline{0,n-1},b_j=\overline{0,m}$ -- неизвестные параметры объекта.
Предполагается, что знак величины $b_m$ известен.
Пусть в решаемой задаче $b_m\geq b_{\min}>0,b_{\min}$ -- известная величина.


Задача --
слежение выходной переменной $y$ за эталонным сигналом $y_M$,
формируемым эталонной моделью вида:
\begin{align}
    y_M(t)=\frac{k_0}{K_M(s)}\left[ g(t) \right],\label{eq:ym}
\end{align}
где $g$ -- сигнал задания, $K_M(s)$ --
гурвицевый полином, определяющий
желаемую динамику замкнутой системы.
Полином $K_M(s)$ строится на
основе метода стандартных полиномов,
исходя из заданных динамических
характеристик.


Цель управления заключается в синтезе управления $u$,
компенсирующего неопределенности объекта и обеспечивающего при
условии ограниченности всех сигналов выполнение целевого равенства:
\begin{align}
    \lim\limits_{t\to\infty}(y_M(t)-y(t))=0\label{eq:aim}
\end{align}


\section{Теоретическая часть}
Для синтеза адаптивного регулятора используется параметризованное
представление выхода объекта:
\begin{align}
    y(t)=\frac{1}{K_M(s)}\left[ \Psi^T\omega(t)+b_mu(t) \right]+\delta(t),\label{eq:yt}
\end{align}
где регрессор $\omega^T=\left[ \nu_1^T,\nu_2^T,y \right],\delta(t)$ --
экспоненциально затухающая функция,
определяемая ненулевыми начальными условиями, $K_M(s)$
-- нормированный устойчивый полином степени $n-m$.


Вместе с моделью рассмотрим динамические фильтры:
\begin{align}
    \dot{\nu}_1=\Lambda \nu_1+e_{n-1}u,\label{eq:dv1}\\
    \dot{\nu}_2=\Lambda \nu_2+e_{n-1}y,\label{eq:dv2}
\end{align}
где $\nu_1,\nu_2\in\mathbb{R}^{n-1}$ -- векторы состояния фильтров,
$e_{n-1}=\operatorname{col}(0,...,0,1),e_{n-1}\in\mathbb{R}^{n-1}$,
$$
\Lambda=\begin{bmatrix}
    0&1&0&...&0\\
    0&0&1&...&0\\
    \vdots&\vdots&\vdots&\ddots&\vdots\\
    0&0&0&...&1\\
    -k_0&-k_1&-k_2&...&-k_{n-2}
\end{bmatrix}
$$
Матрица $\Lambda$ имеет сопровождающий полином:
$$
K(s)=s^{n-1}+k_{n-2}s^{n-2}+k_{n-3}s^{n-3}+...+k_0
$$


Решение задачи адаптивного управления
по выходу предполагает ограниченный класс
объектов вида (\ref{eq:mpomio}). Класс ограничивается строго положительно вещественными (СПВ) передаточными
функциями. Так, например, передаточная функция
$$
H(s)=\frac{1}{K_M(s)}
$$
модели ошибки
\begin{align}
    \varepsilon=y_M-y=\frac{1}{K_M(s)}\left[ \tilde{\Psi}_p^T\omega_p \right],\ \tilde{\Psi}_p^T=\left[ \tilde{\Psi}^T,\tilde{b}_m \right],\ \omega_p^T=\left[ -\omega^T,-u \right]\label{eq:eps7}
\end{align}
с $\tilde{\Psi}=\Psi-\hat{\Psi},\tilde{b}_m=b_m-\hat{b}_m$
при порядке полинома $K_M(s)$ больше единицы не является СПВ, а значит,
алгоритм адаптации
\begin{align}
    \dot{\hat{\Psi}}_p=\gamma\Gamma\omega_p\varepsilon,\ \gamma>0,\ \hat{\Psi}_p^T=\left[ \hat{\Psi}^T,\hat{b}_m \right],\ \Gamma=\begin{cases}
        I_{2n}, &\hat{b}_m(t)\geq b_{\min},\\
        I_{2n}-\zeta_{2n}\zeta_{2n}^T, &\hat{b}_m(t)<b_{\min}
    \end{cases}\label{eq:dhp}
\end{align}
с $\zeta_{2n}=\left[ 0,0,...,0,1 \right]$
не применим.


Для решения этой проблемы
преобразуем динамическую модель
ошибки (\ref{eq:eps7}) к виду:
\begin{align}
    \varepsilon=\tilde{\Psi}_p^T\bar{\omega}_p+\hat{\Psi}_p^T\bar{\omega}_p-\frac{1}{K_M(s)}\left[ \hat{\Psi}_p^T\omega_p \right],\ \bar{\omega}_p=\frac{1}{K_M(s)}\left[ \omega_p \right]\label{eq:eps8}
\end{align}
Введем в рассмотрение сигнал расширенной ошибки:
\begin{align}
    \hat{\varepsilon}=\varepsilon-\hat{\Psi}_p^T\bar{\omega}_p+\frac{1}{K_M(s)}\left[ \hat{\Psi}_p^T\omega_p \right]\label{eq:heps}
\end{align}
С учетом (\ref{eq:eps8}) равенство (\ref{eq:heps}) примет вид:
\begin{align}
    \hat{\varepsilon}=\tilde{\Psi}_p^T\bar{\omega}_p\label{eq:heps2}
\end{align}
Выражение (\ref{eq:heps2})
представляет собой статическую модель ошибки, на
базе которой строится алгоритм адаптации:
\begin{align}
    \dot{\hat{\Psi}}_p=\gamma\Gamma\frac{\bar{\omega}_p}{1+\bar{\omega}_p^T\bar{\omega}_p}\hat{\varepsilon},\label{eq:dhpp}
\end{align}
где $\Gamma$ определена в выражении (\ref{eq:dhp}).


Алгоритм (\ref{eq:dhpp}) с учетом (\ref{eq:dhp}) можно представить в виде:
\begin{align}
    &\dot{\hat{\Psi}}=-\gamma_1\frac{\bar{\omega}}{1+\bar{\omega}^T\bar{\omega}}\hat\varepsilon,\label{eq:dhp2}\\
    &\dot{\hat{b}}_m=\begin{cases}
        -\gamma_2\frac{\bar{u}}{1+\bar{\omega}^T\bar{\omega}}\hat\varepsilon, & \hat{b}_m(t)\geq b_{\min},\\
        0, &\hat{b}_m(t)<b_{\min},
    \end{cases}\label{eq:dhbm}\\
    &u=\frac{1}{\hat{b}_m}\left( -\hat{\Psi}^T\omega+k_0g \right),\label{eq:u}
\end{align}
где
$$
\bar{\omega}=\frac{1}{K_M(s)}\left[ \omega \right],\ \bar{u}=\frac{1}{K_M(s)}\left[ u \right],\ \gamma_{1,2}>0
$$


Таким образом, закон адаптивного управления состоит из
настраиваемого регулятора (\ref{eq:u}), расширенной ошибки (\ref{eq:heps}) и алгоритма
адаптации (\ref{eq:dhpp}). Алгоритм адаптации генерирует настраиваемые
параметры регулятора, содержащиеся в векторе $\hat{\Psi}_p^T$.


Зададим функцию
Ляпунова $V=\tilde{\Psi}_p^T\tilde{\Psi}_p/(2\gamma)$
для анализа устойчивости.
Ее производная:
$$
\dot{V}=\frac{1}{\gamma}\tilde{\Psi}_p^T\dot{\tilde{\Psi}}_p=-\frac{1}{1+\bar{\omega}_p^T\bar{\omega}_p}\hat\varepsilon^2<0
$$
Следовательно, сигналы $\hat\varepsilon,\dot{\hat{\Psi}}_p$ стремятся к нулю
асимптотически. Далее, применяя к (\ref{eq:heps})
Лемму о перестановке, получаем:
\begin{align}
    \hat\varepsilon=\varepsilon-H_{KC}(s)\left[ H_{KB}(s)\left[ \omega_p^T \right]\dot{\hat{\Psi}}_p \right]\label{eq:heps3}
\end{align}
В последнем выражении $H_{KC}(s)=C_K\left( Is-A_K \right)^{-1}$,
$H_{KB}(s)=\left( Is-A_K \right)^{-1}b_K$ --
передаточные матрицы, получаемые на основе модификации
передаточной функции:
$$
\frac{1}{K_M(s)}=C_K\left( Is-A_K \right)^{-1}b_K,
$$
которая рассчитывается на основе тройки матриц
$\left( A_K,b_K,C_K \right)$ --
минимальной реализации передаточной функции $1/K_M(s)$.


Так как $\hat\varepsilon,\dot{\hat{\Psi}}_p$ стремятся к нулю асимптотически, а передаточная
функция $1/K_M(s)$ устойчива, то из выражения (\ref{eq:heps3}) следует сходимость
ошибки управления $\varepsilon$ к нулю асимптотически.


Таким образом, для любых начальных условий $y(0),...,y^{(n-1)}(0),\hat{\Psi}_p(0)$
закон адаптивного управления обеспечивает следующие свойства в
замкнутой системе:
\begin{itemize}
    \item все сигналы в системе ограничены;
    \item ошибка $\varepsilon$ стремится к нулю асимптотически;
    \item параметрические ошибки $\tilde{\Psi}_p$ стремятся к нулю, если вектор $\bar{\omega}_p$
    удовлетворяет условию неисчезающего возбуждения. Это условие
    в конечном итоге зависит от частотной насыщенности сигнала задания $g$,
    который должен содержать <<достаточное>> количество гармоник.
\end{itemize}
\vfill


\section{Экспериментальная часть}
\subsection{Параметры системы}
Согласно варианту 21, исходные данные:
$$
a_0=9,\ a_1=6,\ b_0=9,\ k_{M,1}=6,\ k_{M,0}=9,\ k_0=1,
$$
$$
g(t)=0.4\sin{3t}+\cos{0.1t}
$$


\subsection{Стабилизирующее адаптивное управление}
Построим стабилизирующее адаптивное управление $(g(t)=0)$,
$m=0,\hat{b}_m\equiv\hat{b}_0,b_m\equiv b_0$ на основе
фильтров (\ref{eq:dv1}), (\ref{eq:dv2}), настраиваемого
регулятора (\ref{eq:u}), алгоритма адаптации (\ref{eq:dhp2}), (\ref{eq:dhbm}),
расширенной ошибки (\ref{eq:heps}) и исходных данных. Начальное
условие в алгоритме адаптации $\hat{b}_m(0)>b_0$.


Схема моделирования симулинк:
\begin{figure}[H]
    \centering
    \includegraphics[scale=0.5]{scheme.png}
    \caption{Схема моделирования}
    \label{fig:sch}
\end{figure}
\begin{figure}[H]
    \centering
    \includegraphics[scale=0.4]{sybs.png}
    \caption{Подсистема схемы моделирования}
    \label{fig:sybs}
\end{figure}


Программы, используемые в схеме:
\begin{lstlisting}[label=code1, caption={Код матлаб-функции}]
function [u,nu_dot_1,nu_dot_2,omega,theta_phi, ...
    psi_hat_dot,b0_hat_dot] = fcn(y,e,g,u_bar, ...
    nu_1, nu_2,omega_bar,theta_phi_bar,...
    psi_hat,b0_hat,k0, bmin)
Lambda = -k0;
omega = [nu_1', nu_2', y]';

u = (-psi_hat'*omega + k0*g)/b0_hat;
nu_dot_1 = Lambda*nu_1 + u;
nu_dot_2 = Lambda*nu_2 + y;

theta_hat = [psi_hat; b0_hat];
omega_bar_p = [-omega_bar; -u_bar];
omega_p = [-omega; -u];
theta_phi = theta_hat'*omega_p;
e_hat = e - theta_hat'*omega_bar_p + theta_phi_bar;

gamma1 = 100;
gamma2 = 1;
psi_hat_dot = -gamma1*omega_bar / ...
    (1 + omega_bar'*omega_bar)*e_hat;
b0_hat_dot = 10;
if b0_hat >= bmin
    b0_hat_dot = -gamma2*u_bar / ...
        (1 + omega_bar'*omega_bar)*e_hat;
elseif b0_hat < bmin
    b0_hat_dot = 0;
end
\end{lstlisting}


\begin{lstlisting}[label=code2, caption={Исходные данные схемы}]
a0=9;
a1=6;
b0=9;
bmin=1;
km1=6;
km0=9;
k0=1;
\end{lstlisting}


\begin{lstlisting}[label=code3, caption={Начальные значения интеграторов $\dot{\hat{\Psi}},\dot{\nu}_i,\dot{\hat{b}}_0$}]
set_param('sim1/phd_int','InitialCondition','[0.5;0.5;0.5]');
set_param('sim1/nd1_int','InitialCondition','0.5');
set_param('sim1/nd2_int','InitialCondition','0.5');
set_param('sim1/bhd_int','InitialCondition','10');
set_param('sim1', 'StopTime', '15.0');
\end{lstlisting}


Промоделируем систему (\ref{fig:sch}) при $g(t)=0$ и начальных значениях,
представленных на листинге (\ref{code3}), для различных $\gamma_1\gg \gamma_2$:
\begin{figure}[H]
    \centering
    \includegraphics[scale=1]{1_y_1000_1.png}
    \caption{Выход системы $y(t)$ при $\gamma_1=1000,\gamma_2=1$}
    \label{fig:1_y_1000_1}
\end{figure}
\begin{figure}[H]
    \centering
    \includegraphics[scale=1]{1_u_1000_1.png}
    \caption{Управление $u(t)$ при $\gamma_1=1000,\gamma_2=1$}
    \label{fig:1_u_1000_1}
\end{figure}
\begin{figure}[H]
    \centering
    \includegraphics[scale=1]{1_p_1000_1.png}
    \caption{Оценки параметров $\hat{\Psi}_p(t)$ при $\gamma_1=1000,\gamma_2=1$}
    \label{fig:1_p_1000_1}
\end{figure}


\begin{figure}[H]
    \centering
    \includegraphics[scale=1]{1_y_10000_10.png}
    \caption{Выход системы $y(t)$ при $\gamma_1=10000,\gamma_2=10$}
    \label{fig:1_y_10000_10}
\end{figure}
\begin{figure}[H]
    \centering
    \includegraphics[scale=1]{1_u_10000_10.png}
    \caption{Управление $u(t)$ при $\gamma_1=10000,\gamma_2=10$}
    \label{fig:1_u_10000_10}
\end{figure}
\begin{figure}[H]
    \centering
    \includegraphics[scale=1]{1_p_10000_10.png}
    \caption{Оценки параметров $\hat{\Psi}_p(t)$ при $\gamma_1=10000,\gamma_2=10$}
    \label{fig:1_p_10000_10}
\end{figure}


\begin{figure}[H]
    \centering
    \includegraphics[scale=1]{1_y_100000_100.png}
    \caption{Выход системы $y(t)$ при $\gamma_1=100000,\gamma_2=100$}
    \label{fig:1_y_100000_100}
\end{figure}
\begin{figure}[H]
    \centering
    \includegraphics[scale=1]{1_u_100000_100.png}
    \caption{Управление $u(t)$ при $\gamma_1=100000,\gamma_2=100$}
    \label{fig:1_u_100000_100}
\end{figure}
\begin{figure}[H]
    \centering
    \includegraphics[scale=1]{1_p_100000_100.png}
    \caption{Оценки параметров $\hat{\Psi}_p(t)$ при $\gamma_1=100000,\gamma_2=100$}
    \label{fig:1_p_100000_100}
\end{figure}


При увеличении коэффициентов $\gamma_i$ выход системы
быстрее сходится к нулю.


Оценки параметров не сходятся к истинным значениям, так как
условие неисчезающего возбуждения для $\bar{\omega}_p$ при $g(t)=0$ не выполняется.


\subsection{Следящий адаптивный регулятор: устойчивая система}
На основе эталонной модели (\ref{eq:ym}), фильтров (\ref{eq:dv1}), (\ref{eq:dv2}),
настраиваемого регулятора (\ref{eq:u}), алгоритма адаптации (\ref{eq:dhpp}) и исходных данных,
построим следящий адаптивный
регулятор.


Используемые в схеме программы:
\begin{lstlisting}[label=code, caption={Матлаб-функция задающего воздействия}]
function g = set_signal(t)
g = 0.4*sin(3*t)+cos(0.1*t);
\end{lstlisting}


Промоделируем систему (\ref{fig:sch}) при $g(t)\neq0$, $\dot{\hat{\Psi}}(0)=\left[ 0;0;0 \right]$,
$\dot{\nu}_i(0)=0$, $\dot{\hat{b}}_0(0)=10$ для различных $\gamma_1\gg \gamma_2$:
\begin{figure}[H]
    \centering
    \includegraphics[scale=0.7]{2_y_100000_100.png}
    \caption{Выход сигнала $y_M(t)$ и системы $y(t)$ при $\gamma_1=100000,\gamma_2=100$}
    \label{fig:2_y_100000_100}
\end{figure}
\begin{figure}[H]
    \centering
    \includegraphics[scale=0.7]{2_u_100000_100.png}
    \caption{Управление $u(t)$ при $\gamma_1=100000,\gamma_2=100$}
    \label{fig:2_u_100000_100}
\end{figure}
\begin{figure}[H]
    \centering
    \includegraphics[scale=1]{2_p_100000_100.png}
    \caption{Оценки параметров $\hat{\Psi}_p(t)$ при $\gamma_1=100000,\gamma_2=100$}
    \label{fig:2_p_100000_100}
\end{figure}


\begin{figure}[H]
    \centering
    \includegraphics[scale=0.7]{2_y_1000000_1000.png}
    \caption{Выход сигнала $y_M(t)$ и системы $y(t)$ при $\gamma_1=1000000,\gamma_2=1000$}
    \label{fig:2_y_1000000_1000}
\end{figure}
\begin{figure}[H]
    \centering
    \includegraphics[scale=0.7]{2_u_1000000_1000.png}
    \caption{Управление $u(t)$ при $\gamma_1=1000000,\gamma_2=1000$}
    \label{fig:2_u_1000000_1000}
\end{figure}
\begin{figure}[H]
    \centering
    \includegraphics[scale=1]{2_p_1000000_1000.png}
    \caption{Оценки параметров $\hat{\Psi}_p(t)$ при $\gamma_1=1000000,\gamma_2=1000$}
    \label{fig:2_p_1000000_1000}
\end{figure}
\begin{figure}[H]
    \centering
    \includegraphics[scale=1]{2_p_1000000_1000_close.png}
    \caption{Оценки параметров $\hat{\Psi}(t)$ вблизи при $\gamma_1=1000000,\gamma_2=1000$}
    \label{fig:2_p_1000000_1000_close}
\end{figure}


\begin{figure}[H]
    \centering
    \includegraphics[scale=0.7]{2_y_5000000_5000.png}
    \caption{Выход сигнала $y_M(t)$ и системы $y(t)$ при $\gamma_1=5000000,\gamma_2=5000$}
    \label{fig:2_y_5000000_5000}
\end{figure}
\begin{figure}[H]
    \centering
    \includegraphics[scale=0.7]{2_u_5000000_5000.png}
    \caption{Управление $u(t)$ при $\gamma_1=5000000,\gamma_2=5000$}
    \label{fig:2_u_5000000_5000}
\end{figure}
\begin{figure}[H]
    \centering
    \includegraphics[scale=1]{2_p_5000000_5000.png}
    \caption{Оценки параметров $\hat{\Psi}_p(t)$ при $\gamma_1=5000000,\gamma_2=5000$}
    \label{fig:2_p_5000000_5000}
\end{figure}
\begin{figure}[H]
    \centering
    \includegraphics[scale=1]{2_p_5000000_5000_close.png}
    \caption{Оценки параметров $\hat{\Psi}(t)$ вблизи при $\gamma_1=5000000,\gamma_2=5000$}
    \label{fig:2_p_5000000_5000_close}
\end{figure}


Чем больше $\gamma_i$, тем быстрее выход системы
сходится к желаемому сигналу, быстрее происходит адаптация параметров.


Оценки параметров сходятся к истинным значениям, так как
условие неисчезающего возбуждения для $\bar{\omega}_p$
выполняется.


\subsection{Следящий адаптивный регулятор: неустойчивая система}
Выберем коэффициенты модели объекта $a_i$ с обратным знаком:
$$
a_1=-6,\ a_0=-9
$$


Проведем моделирование системы (\ref{fig:sch}) при
таких же начальных условиях, как и в предыдущем пункте,
для одной пары $\gamma_1,\gamma_2$:
\begin{figure}[H]
    \centering
    \includegraphics[scale=0.7]{3_y_1000_1.png}
    \caption{Выход сигнала $y_M(t)$ и системы $y(t)$ при $\gamma_1=1000,\gamma_2=1$}
    \label{fig:3_y_1000_1}
\end{figure}
\begin{figure}[H]
    \centering
    \includegraphics[scale=0.7]{3_u_1000_1.png}
    \caption{Управление $u(t)$ при $\gamma_1=1000,\gamma_2=1$}
    \label{fig:3_u_1000_1}
\end{figure}
\begin{figure}[H]
    \centering
    \includegraphics[scale=1]{3_p_1000_1.png}
    \caption{Оценки параметров $\hat{\Psi}_p(t)$ при $\gamma_1=1000,\gamma_2=1$}
    \label{fig:3_p_1000_1}
\end{figure}


Управления стало затрачиваться значительно больше,
в системе присутствует перерегулирование.


Оценки параметров не сходятся к истинным значениям.


\section{Вывод}
В ходе выполнения данной лабораторной
работы был синтезирован следящий адаптивный регулятор
на основе метода расширенной ошибки.
В случае устойчивого объекта и при выполнении
условия неисчезающего возбуждения оценки
параметров сошлись к истинным значениям.
Было проведено исследование влияния параметров
$\gamma_1,\gamma_2:\gamma_1\gg \gamma_2$ на работу адаптивного алгоритма.
Чем больше их значения, тем быстрее адаптация параметров
и схождение выхода системы к эталонному сигналу.
В случае неустойчивого объекта или при невыполнении
условия неисчезающего возбуждения оценки параметров
не сошлись к истинным значениям.
\end{document}