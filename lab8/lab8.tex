\documentclass[a4paper,14pt]{extarticle}

\usepackage[T2A]{fontenc}
\usepackage[utf8]{inputenc}
\usepackage[english, russian]{babel}

\usepackage[left=30mm, right=10mm, top=20mm, bottom=20mm]{geometry}

\usepackage{tempora}
\usepackage{setspace}
\onehalfspacing

\usepackage{titlesec}
\titleformat{\section}[block]{\bfseries\centering\MakeUppercase}{\thesection.}{1em}{}
\titleformat{\subsection}[block]{\bfseries}{\thesubsection.}{1em}{}
\titleformat{\subsubsection}[block]{\bfseries}{\thesubsubsection.}{1em}{}

\renewcommand{\contentsname}{\hfill \textbf{СОДЕРЖАНИЕ} \hfill\null}

\usepackage{indentfirst}
\setlength{\parindent}{1.25cm}

\usepackage{amsmath, amsfonts, amssymb}
\usepackage{graphicx}
\usepackage{caption}
\usepackage{subcaption}
\usepackage{float}
\usepackage{tikz}
\usetikzlibrary{patterns}
\usepackage{cmap}
\usepackage{hyperref}
\usepackage{xcolor}
\usepackage{listings}

\definecolor{LightGray}{gray}{0.7}

\lstdefinestyle{code}{
    language=matlab, % change if needed
    basicstyle=\small\ttfamily,
    numbers=left,
    numberstyle=\small\color{LightGray},
    stepnumber=1,
    numbersep=5pt,
    backgroundcolor=\color{white},
    showspaces=false,
    showstringspaces=false,
    showtabs=false,
    tabsize=4,
    captionpos=b,
    breaklines=true,
    breakatwhitespace=false,
    frame=single,
    rulecolor=\color{LightGray},
    linewidth=\linewidth,
    keywordstyle=\color{blue}\bfseries,
    commentstyle=\color{green!40!black},
    stringstyle=\color{violet},
    escapeinside={\%*}{*)},
    xleftmargin=10pt,
    xrightmargin=10pt,
    framexleftmargin=0pt,
    framexrightmargin=0pt
}
\lstset{style=code}

\hypersetup{
    colorlinks=true,
    linkcolor=blue,
    filecolor=magenta,
    urlcolor=cyan,
    pdftitle={lab arc},
    pdfauthor={Rumyantsev Alexey},
    pdfsubject={control},
    pdfkeywords={LaTeX, PDF},
    pdfpagemode=FullScreen,
}

\graphicspath{{src/images/}}

\begin{document}

\begin{titlepage}
    \begin{center}
        МИНИСТЕРСТВО НАУКИ И ВЫСШЕГО ОБРАЗОВАНИЯ РОССИЙСКОЙ ФЕДЕРАЦИИ\\
        \vspace*{2.5mm}
        Федеральное государственное автономное образовательное учреждение высшего образования
        «НАЦИОНАЛЬНЫЙ ИССЛЕДОВАТЕЛЬСКИЙ УНИВЕРСИТЕТ ИТМО»\\
        \vspace*{2.5mm}
        ФАКУЛЬТЕТ СИСТЕМ УПРАВЛЕНИЯ И РОБОТОТЕХНИКИ
        \vfill

        {\large\bfseries ОТЧЕТ ПО ЛАБОРАТОРНОЙ РАБОТЕ №8}\\
        {\large по дисциплине}\\
        {\large«АДАПТИВНОЕ И РОБАСТНОЕ УПРАВЛЕНИЕ»}\\
        {\large на тему}\\
        {\large «АДАПТИВНОЕ УПРАВЛЕНИЕ ЛИНЕЙНЫМ
                ОБЪЕКТОМ ПО ВЫХОДУ НА ОСНОВЕ АЛГОРИТМА АДАПТАЦИИ С
                РАСШИРЕННОЙ ОШИБКОЙ»}\\
        Вариант 21
        \vfill

        \begin{flushright}
            Выполнили: студенты\\
            Дьячихин Д. Н., R3480\\
            Румянцев А. А., R3441\medskip\\

            Проверил: преподаватель\\
            Парамонов А. В.
        \end{flushright}
        \vfill

        Санкт-Петербург\\
        2025
    \end{center}
\end{titlepage}

\setcounter{page}{2}
\tableofcontents
\newpage

\section{Цель работы}
Освоение метода расширенной ошибки в задачах
адаптивного управления по выходу.


\section{Постановка задачи}
Рассмотрим минимально-фазовую
линейную модель объекта,
представленную в форме <<вход-выход>>:
\begin{align}
    y^{(n)}+a_{n-1}y^{(n-1)}+a_{n-2}y^{(n-2)}+...+a_0y=b_mu^{(m)}+b_{m-1}u^{(m-1)}+...+b_0u,\label{eq:mpomio}
\end{align}
где $a_i,i=\overline{0,n-1},b_j=\overline{0,m}$ -- неизвестные параметры объекта.
Предполагается, что знак величины $b_m$ известен. Пусть в решаемой задаче
$b_m\geq b_{\min}>0,b_{\min}$ -- известная величина.


Вместе с моделью рассмотрим динамические фильтры:
\begin{align}
    \dot{\nu}_1=\Lambda \nu_1+e_{n-1}u,\label{eq:dv1}\\
    \dot{\nu}_2=\Lambda \nu_2+e_{n-1}y,\label{eq:dv2}
\end{align}
где $\nu_1,\nu_2\in\mathbb{R}^{n-1}$ -- векторы состояния фильтров,
$e_{n-1}=\operatorname{col}(0,...,0,1),e_{n-1}\in\mathbb{R}^{n-1}$,
$$
\Lambda=\begin{bmatrix}
    0&1&0&...&0\\
    0&0&1&...&0\\
    \vdots&\vdots&\vdots&\ddots&\vdots\\
    0&0&0&...&1\\
    -k_0&-k_1&-k_2&...&-k_{n-2}
\end{bmatrix}
$$
Матрица $\Lambda$ имеет сопровождающий полином:
$$
K(s)=s^{n-1}+k_{n-2}s^{n-2}+k_{n-3}s^{n-3}+...+k_0
$$
Для любых нормированных устойчивых полиномов
$K(s),K_M(s)$ степени $n-1,n-m$ соответственно
существует единственный постоянный вектор
$\Psi\in\mathbb{R}^{2n-1}$, зависящий от неизвестных
параметров объекта, такой, что объект (\ref{eq:mpomio})
может быть представлен в виде:
\begin{align}
    y(t)=\frac{1}{K_M(s)}\left[ \Psi^T\omega(t)+b_mu(t) \right]+\delta(t),\label{eq:yt}
\end{align}
где $\omega^T=\left[ \nu_1^T,\nu_2^T,y \right],\delta(t)$ --
экспоненциально затухающая функция,
определяемая ненулевыми начальными условиями.


Параметризованное представление (\ref{eq:yt}) позволяет синтезировать
управление, компенсирующее неопределенности модели, сосредоточенные
в векторе $\Psi$.


Рассмотрим задачу
слежения выходной переменной $y$ за эталонным сигналом $y_M$,
формируемым эталонной моделью вида:
\begin{align}
    y_M(t)=\frac{k_0}{K_M(s)}\left[ g(t) \right],\label{eq:ym}
\end{align}
где $g$ -- сигнал задания, $K_M(s)$ --
гурвицевый полином, определяющий
желаемую динамику замкнутой системы.
Полином $K_M(s)$ строится на
основе метода стандартных полиномов,
исходя из заданных динамических
характеристик.


Цель управления заключается в синтезе управления $u$,
компенсирующего неопределенности объекта и обеспечивающего при
условии ограниченности всех сигналов выполнение целевого равенства:
\begin{align}
    \lim\limits_{t\to\infty}(y_M(t)-y(t))=0\label{eq:aim}
\end{align}


\section{Теоретическая часть}
% Для решения задачи сформируем ошибку управления по выходу $\varepsilon=y_M-y$
% и с учетом (\ref{eq:yt}), (\ref{eq:ym}), проведем простейшие преобразования:
% \begin{align*}
%     \varepsilon=&\frac{k_0}{K_M(s)}[g]-\frac{1}{K_M(s)}\left[ \Psi^T\omega+b_m u \right]=\\
%     &=\frac{1}{K_M(s)}\left[ k_0g-\Psi^T\omega-b_mu \right]=\frac{1}{K_M(s)}\left[ k_0g-\Psi^T\omega-b_mu \right]
% \end{align*}


% Окончательно имеем:
% \begin{align}
%     \varepsilon=\frac{1}{K_M(s)}\left[ k_0g-\Psi^T\omega-b_mu \right]\label{eq:eps}
% \end{align}


% Выражение (\ref{eq:eps}) позволяет сформировать компенсирующий закон
% управления вида:
\subsection{Способ №1 ($\boldsymbol{b_m}$ известно)}
Закон управления формируется в виде
\begin{align}
    u=\frac{1}{b_m}\left( \hat{\Psi}^T\omega_p+k_0g \right)\label{eq:u}
\end{align}
% \begin{align}
%     u=\frac{1}{\hat{b}_m}\left( -\hat{\Psi}^T\omega_p+k_0g \right),\label{eq:u2}
% \end{align}
% где $\hat{\Psi}$ -- вектор оценок $\Psi$, $\hat{b}_m$ -- оценка $b_m$.
Введем в рассмотрение сигнал расширенной ошибки:
\begin{align}
    \hat{\varepsilon}=\varepsilon-\hat{\Psi}^T\overline{\omega}_p+\frac{1}{K_M(s)}\left[ \hat{\Psi}^T\omega_p \right],\label{eq:heps}
\end{align}
где:
$$\omega_p=-\omega,\ \overline{\omega}_p=\frac{1}{K_M(s)}\left[ \omega_p \right]$$
Тогда, с учетом:
\begin{align}
    \varepsilon=\tilde{\Psi}_p^T\overline{\omega}_p+\hat{\Psi}_p^T\overline{\omega}_p-\frac{1}{K_M(s)}\left[ \hat{\Psi}_p^T\omega_p \right]\label{eq:eps3}
\end{align}
(\ref{eq:heps}) примет вид:
\begin{align}
    \hat{\varepsilon}=\tilde{\Psi}^T\overline{\omega}_p\label{eq:heps2}
\end{align}
Выражение (\ref{eq:heps2}) представляет собой статическую модель ошибки, на
базе которой строится алгоритм адаптации:
\begin{align}
    \dot{\hat{\Psi}}=\gamma\frac{\overline{\omega}_p}{1+\overline{\omega}_p^T\overline{\omega}_p}\hat{\varepsilon}\label{eq:dhpsi}
\end{align}


% С целью предотвращения деления на ноль в законе
% управления (\ref{eq:u}) необходимо использовать алгоритм адаптации с
% добавлением оператора проекции.


% Для формирования алгоритма адаптации, генерирующего оценки $\hat{\Psi}$
% и $\hat{b}_m$, подставим (\ref{eq:u}) в (\ref{eq:eps})
% и получим динамическую модель ошибок с
% измеряемым выходом:
% \begin{align}
%     \varepsilon=\frac{1}{K_M(s)}\left[ \tilde{\Psi}^T\omega-\tilde{b}_mu \right]=\frac{1}{K_M(s)}\left[ \tilde{\Psi}_p^T\omega_p \right],\label{eq:eps2}
% \end{align}
% где $\tilde{\Psi}=\Psi-\hat{\Psi},\tilde{b}_m=b_m-\hat{b}_m$ --
% параметрические ошибки, $\tilde{\Psi}_p^T=\left[ \tilde{\Psi}^T,\tilde{b}_m \right],
% \omega_p^T=\left[ -\omega^T,-u \right]$.


% В случае, если передаточная функция:
% $$
% H(s)=\frac{1}{K_M(s)}
% $$
% является строго положительно вещественной (СПВ), алгоритм адаптации
% для настройки регулятора (\ref{eq:u}) может быть представлен в следующей
% форме:
% \begin{align}
%     \dot{\hat{\Psi}}_p=\gamma\Gamma\omega_p\varepsilon,\label{eq:dhp}
% \end{align}
% где $\gamma>0$ -- коэффициент адаптации, $\hat{\Psi}_p^T=\left[ \hat{\Psi}^T,\hat{b}_m \right]$,
% \begin{align}
%     \Gamma=\begin{cases}
%         I_{2n}, &\hat{b}_m(t)\geq b_{\min},\\
%         I_{2n}-\zeta_{2n}\zeta_{2n}^T, &\hat{b}_m(t)<b_{\min},
%     \end{cases}\label{eq:Gamma}
% \end{align}
% где $\zeta_{2n}=\left[ 0,0,...,0,1 \right]$ --
% координатный вектор размерности $2n$.
% Вторая строка в последнем выражении позволяет
% <<остановить>> функцию $\hat{b}_m(t)$
% в момент пересечения границы $b_{\min}$
% в целях избежать деление на ноль в (\ref{eq:u}).


% С помощью выражений (\ref{eq:dhp}), (\ref{eq:Gamma}) можно показать, что оценки $\hat{\Psi}$
% и $\hat{b}_m$, необходимые для закона управления (\ref{eq:u}), генерируются согласно
% следующим правилам:
% \begin{align}
%     &\dot{\hat{\Psi}}=-\gamma\omega\varepsilon,\label{eq:dhp2}\\
%     &\dot{\hat{b}}_m=\begin{cases}
%         -\gamma u\varepsilon, &\hat{b}_m(t)\geq b_{\min},\\
%         0, &\hat{b}_m(t)<b_{\min}
%     \end{cases}\label{eq:dhb}
% \end{align}


% Таким образом, закон адаптивного управления, построенный на
% основе параметризованного представления (\ref{eq:yt}), состоит из эталонной
% модели (\ref{eq:ym}), настраиваемого регулятора (\ref{eq:u}) и алгоритмов адаптации
% (\ref{eq:dhp2}), (\ref{eq:dhb}). Закон управления формируется на основе измерения
% выходной переменной и не использует информацию о состоянии объекта,
% что является его отличительной особенностью.


% Важно отметить, что начальное условие $\hat{b}_m(0)$ в алгоритме (\ref{eq:dhb})
% выбирается из условия $\hat{b}_m(0)\geq b_{\min}$.


\subsection{Способ №2 ($\boldsymbol{b_m}$ неизвестно)}
Закон управления формируется в виде:
\begin{align}
    u=\hat{\Psi}^T\omega\label{eq:u3}
\end{align}
Введем в рассмотрение сигнал расширенной ошибки:
\begin{align}
    \hat{\varepsilon}=\varepsilon+\hat{k}\xi,\label{eq:heps3}
\end{align}
где:
$$
\xi=\frac{1}{K_M(s)}\left[ \hat{\Psi}^T\omega \right]-\hat{\Psi}^T\overline{\omega},\ \omega=\left[ \nu_1,\nu_2,y,g \right],\ \overline{\omega}\frac{1}{K_M(s)}\left[ \omega \right]
$$
Алгоритм адаптации примет следующий вид:
\begin{align}
    \dot{\hat{\Psi}}=\gamma_1\frac{\overline{\omega}}{1+\overline{\omega}^T\overline{\omega}}\hat{\varepsilon},\label{eq:dhpsi2}\\
    \dot{\hat{k}}=-\gamma_2\frac{\xi}{1+\overline{\omega}^T\overline{\omega}}\hat{\varepsilon}\label{eq:dhk}
\end{align}


\section{Экспериментальная часть}
\subsection{Параметры системы}
Согласно варианту 21, исходные данные:
$$
a_0=9,\ a_1=6,\ b_0=9,\ k_{M,1}=6,\ k_{M,0}=9,\ k_0=1,\ g(t)=0.4\sin{3t}+\cos{0.1t}
$$


\section{Вывод}
...
\end{document}